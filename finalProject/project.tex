\documentclass[11pt]{article}

%%%%%%%%%%%%%%% PACKAGES %%%%%%%%%%%%%%%

\usepackage{url}
\usepackage{amssymb}
\usepackage{multicol}
\usepackage{enumerate}
\usepackage{tabularx}
\usepackage{graphicx}
\usepackage{amsmath}
\usepackage{amsfonts}
\usepackage{amsthm}
\usepackage{amscd}
\usepackage{amsrefs}
\usepackage[affil-it]{authblk}
\usepackage{mathtools}

%%%%%%%%%%%%%%% TITLE/AUTHOR INFORMATION %%%%%%%%%%%%%%%

\title{Digital Signal Processing}
\author{Marvin Lin}
\date{\today}
\affil{Portland Community College}

%%%%%%%%%%%%%%% CUSTOM FORMAT %%%%%%%%%%%%%%%

\setlength{\textwidth}{6.5in}
\setlength{\textheight}{9in}
\setlength{\evensidemargin}{0in}
\setlength{\oddsidemargin}{0in}
\setlength{\topmargin}{+.4in}

\setlength{\oddsidemargin}{0.in}
\setlength{\evensidemargin}{0.in}
\setlength{\textwidth}{6.46in}
\setlength{\textheight}{8.4in}

\usepackage[margin=2.5cm]{geometry}

%%%%%%%%%%%%%%% PREAMBLE %%%%%%%%%%%%%%%

\newcommand{\Z}{\mathbb{Z}}
\newcommand{\Q}{\mathbb{Q}}
\newcommand{\R}{\mathbb{R}}
\newcommand{\C}{\mathbb{C}}
\DeclareMathOperator{\rank}{rank}               % Rank
\DeclareMathOperator{\vol}{vol}                 % Volume
\DeclarePairedDelimiter\abs{\lvert}{\rvert}     % Absolute Value

%%%%%%%%%%%%%%% BEGIN DOCUMENT %%%%%%%%%%%%%%%

\begin{document}

%%%%%%%%%%%%%%% TITLE %%%%%%%%%%%%%%%

\maketitle

%%%%%%%%%%%%%%% ABSTRACT %%%%%%%%%%%%%%%

\begin{abstract}

We give an example of what a paper might look like and means to be a template for the student's paper. \TeX\ is not easy to learn, so we streamline the process with this template. Any technical questions should be Google'd and/or posed to the instructor.
% An abstract is a short, one-paragraph summary of what your paper is about. 
% It should not have any technical details and will act like the back-cover of a novel that summarizes what you are going to read if you read the paper. 
% You must use the abstract environment. 
% Do not refer to your paper. That is, don't say, ``In the paper, we discuss...'' or ``In our paper, we discuss...''. Just say ``We discuss...''

\end{abstract}

%%%%%%%%%%%%%%% BODY OF PAPER %%%%%%%%%%%%%%%

For example, I may have written a book. \cite{book:author}. Or I may have viewed a website. \cite{website:topic}.

%%%%%%%%%%%%%%% BIBLIOGRAPHY %%%%%%%%%%%%%%%

\bibliographystyle{plain} % We choose the "plain" reference style
\bibliography{refs} % Entries are in the "refs.bib" file

%%%%%%%%%%%%%%% END DOCUMENT %%%%%%%%%%%%%%%

\end{document} 