\documentclass{report}

\usepackage{marvin} %found on https://github.com/marvinlinn/latexSetup. Add files to your local MiKTeX environment in (MiKTex Console > Settings > Directories > Add) and add the latexSetup folder.

\title{Winter 2022 MTH 261 Mini Test 3}
\author{Marvin Lin}
\date{March 2022}

\begin{document}

\maketitle

\section*{Question 1}
\begin{example}
    Let $A = \begin{bmatrix} 2 & 6 & 14 & 8 \\ 1 & 0 & 0 & 2 \\ -1 & -3 & -5 & 4 \\ -1 & 0 & -7 & 0 \end{bmatrix}$. 
    \begin{enumerate}
    \item Compute $\det A$. 
    \item Determine if $A$ is invertible or singular. Justify your answer as specifically as possible.
    \item Are the vectors $\begin{bmatrix} 2 \\ 1 \\ -1 \\ -1 \end{bmatrix}, \begin{bmatrix} 14 \\ 0 \\ -5 \\ -7 \end{bmatrix}, \begin{bmatrix} 6 \\ 0 \\ -3 \\ 0 \end{bmatrix}, \begin{bmatrix} 8 \\ 2 \\ 4 \\ 0 \end{bmatrix}$ linearly independent? Why or why not?
    \item Does the set of vectors $\left\{\begin{bmatrix} 2 \\ 1 \\ -1 \\ -1 \end{bmatrix}, \begin{bmatrix} 14 \\ 0 \\ -5 \\ -7 \end{bmatrix}, \begin{bmatrix} 6 \\ 0 \\ -3 \\ 0 \end{bmatrix}, \begin{bmatrix} 8 \\ 2 \\ 4 \\ 0 \end{bmatrix}\right\}$ span $\mathbb{R}^4$? Why or why not?
    \end{enumerate}
\end{example}
\subsubsection*{Part 1}
$\begin{vmatrix}
    2 & 6 & 14 & 8 \\ 
    1 & 0 & 0 & 2 \\ 
    -1 & -3 & -5 & 4 \\ 
    -1 & 0 & -7 & 0
\end{vmatrix}$\\\vspace{3mm}
$=-\begin{vmatrix}
    1 & 0 & 0 & 2 \\
    2 & 6 & 14 & 8 \\ 
    -1 & -3 & -5 & 4 \\ 
    -1 & 0 & -7 & 0
\end{vmatrix}$\\\vspace{3mm}
$=-\begin{vmatrix}
    1 & 0 & 0 & 2 \\
    0 & 6 & 14 & 4 \\ 
    0 & -3 & -5 & 6 \\ 
    0 & 0 & -7 & 2
\end{vmatrix}$\\\vspace{3mm}
$=-2\begin{vmatrix}
    1 & 0 & 0 & 2 \\
    0 & 3 & 7 & 2 \\ 
    0 & -3 & -5 & 6 \\ 
    0 & 0 & -7 & 2
\end{vmatrix}$\\\vspace{3mm}
$=-2\begin{vmatrix}
    1 & 0 & 0 & 2 \\
    0 & 3 & 7 & 2 \\ 
    0 & 0 & 2 & 8 \\ 
    0 & 0 & -7 & 2
\end{vmatrix}$\\\vspace{3mm}
$=-4\begin{vmatrix}
    1 & 0 & 0 & 2 \\
    0 & 3 & 7 & 2 \\ 
    0 & 0 & 1 & 4 \\ 
    0 & 0 & -7 & 2
\end{vmatrix}$\\\vspace{3mm}
$=-4\begin{vmatrix}
    1 & 0 & 0 & 2 \\
    0 & 3 & 7 & 2 \\ 
    0 & 0 & 1 & 4 \\ 
    0 & 0 & 0 & 30
\end{vmatrix}$\\\vspace{3mm}
$=-4(1)(3)(1)(30)$\\\vspace{3mm}
$=-360$

\subsubsection*{Part 2}
Since det $A$ is not equal to 0, we know that $A$ is invertible by the Invertible Matrix Theorem (IMT).

\subsubsection*{Part 3}
By the Invertible Matrix Theorem, we know that because $A$ is invertible, the columns of A are also linearly independent, which happens to be the matrix we took the determinant of. Therefore, we know that the vectors are linearly independent.

\subsubsection*{Part 4}
Yes, it does span $\mathbb{R}^4$, as the invertible matrix theorem states that if the matrix $A$ with size $n\times n$ is invertible, then the matrix also spans $\mathbb{R}^4$.
\clearpage

\section*{Question 2}
\begin{example}
    Let $D$ be the set of all diagonal $2 \times 2$ matrices. That is
    $$D = \left\{ \begin{bmatrix} a & 0 \\ 0 & b \end{bmatrix} \bigg|\ a, b \in \mathbb{R} \right\}.$$
    
    Use the subspace test to prove that $D$ is a subspace of $M_{2 \times 2}$.
\end{example}
Most notably, we can see that $D$ is a subset of $M_{2\times2}$\\

First, we can prove that the set is not empty. We can determine that the set is not empty as $\begin{bmatrix} a & 0 \\ 0 & b \end{bmatrix}$ exists in the set, and both $a$ and $b$ are real numbers.\\

Second, we can determine that the set is closed under vector addition:
\begin{center}
    $A_1 = \begin{bmatrix} a_1 & 0 \\ 0 & b_2 \end{bmatrix}$\\
    $A_2 = \begin{bmatrix} a_2 & 0 \\ 0 & b_2 \end{bmatrix}$\\
    $A_1 + A_2 = \begin{bmatrix} a_1 & 0 \\ 0 & b_1 \end{bmatrix} + \begin{bmatrix} a_2 & 0 \\ 0 & b_2 \end{bmatrix} = \begin{bmatrix} a_1+a_2 & 0 \\ 0 & b_1+b_2 \end{bmatrix} \epsilon M_{2\times2}$\\
\end{center}

Lastly, we can determine that the set is closed under scalar multiplication:
\begin{center}
    $cA = c\begin{bmatrix} a & 0 \\ 0 & b \end{bmatrix}$\\
    $= c\begin{bmatrix} ca & 0 \\ 0 & cb \end{bmatrix}\epsilon M_{2\times2}$\\
\end{center}
Therefore, it can be determined that the $D$ is a subspace of $M_{2\times 2}$ by the subspace test.
\clearpage

\section*{Question 3}
\begin{example}
    Let $A = \begin{bmatrix} 1 & 1 & -1 \\ 1 & -1 & 1 \\ -1 & 1 & 1 \end{bmatrix}$.
    \begin{enumerate}
     \item What does it mean for $\mathbf{x}$ to be in $Col A$? 
     \item What does it mean for $\mathbf{x}$ to be in $Nul A$?
     \item Find a spanning set for $Col A$.
     \item Find a spanning set for $Nul A$.
    \end{enumerate}
\end{example}

\subsubsection*{Part 1}

For $\mathbf{x}$ to be in $Col A$, it means that the vector exists as some linear combination of the spanning set of for $Col A$, which is simply the columns of $A$.

\subsubsection*{Part 2}

For $\mathbf{x}$ to be in $Nul A$, it means that the vector exists as some linear combination of the spanning set of for $Nul A$, which can be found by solving the homogenous equation and then using the vectors corresponding to $\mathbf{x}$ in parametric vector form.

\subsubsection*{Part 3}
The spanning set is simply the set of all of the columns in the matrix $A$. Therefore:
\begin{center}
    $\left\{
        \begin{bmatrix} 1 \\ 1 \\ -1 \end{bmatrix}
        \begin{bmatrix} 1 \\ -1 \\ 1 \end{bmatrix}
        \begin{bmatrix} -1 \\ 1 \\ 1 \end{bmatrix}
    \right\}$ 
\end{center}
is the spanning set.
\subsubsection*{Part 4}
\begin{center}
    $A\mathbf{x}=\mathbf{0}$\\
    $\begin{bmatrix} 1 & 1 & -1 \\ 1 & -1 & 1 \\ -1 & 1 & 1 \end{bmatrix}\begin{bmatrix}
        x_1 \\
        x_2 \\
        x_3
    \end{bmatrix} = 
    \begin{bmatrix}
        0 \\
        0 \\
        0
    \end{bmatrix}$
\end{center}

Creating an augmented matrix, we get:

\begin{center}
    $\begin{bmatrix} 
        1 & 1 & -1 & 0\\ 
        1 & -1 & 1 & 0\\ 
        -1 & 1 & 1 & 0
    \end{bmatrix}$\\\vspace{3mm}
    $\begin{bmatrix} 
        1 & 1 & -1 & 0\\ 
        0 & -2 & 2 & 0\\ 
        0 & 2 & 0 & 0
    \end{bmatrix}$\\\vspace{3mm}
    $\begin{bmatrix} 
        1 & 1 & -1 & 0\\ 
        0 & -2 & 2 & 0\\ 
        0 & 0 & 2 & 0
    \end{bmatrix}$\\\vspace{3mm}
    $\begin{bmatrix} 
        1 & 1 & -1 & 0\\ 
        0 & 1 & -1 & 0\\ 
        0 & 0 & 1 & 0
    \end{bmatrix}$\\\vspace{3mm}
    $\begin{bmatrix} 
        1 & 1 & -1 & 0\\ 
        0 & 1 & 0 & 0\\ 
        0 & 0 & 1 & 0
    \end{bmatrix}$\\\vspace{3mm}
    $\begin{bmatrix} 
        1 & 1 & 0 & 0\\ 
        0 & 1 & 0 & 0\\ 
        0 & 0 & 1 & 0
    \end{bmatrix}$\\\vspace{3mm}
    $\begin{bmatrix} 
        1 & 0 & 0 & 0\\ 
        0 & 1 & 0 & 0\\ 
        0 & 0 & 1 & 0
    \end{bmatrix}$\\\vspace{3mm}
    Given the identity matrix after row reduction, we know that the resulting $\mathbf{x}$ is simply $\mathbf{0}$. Therefore, the spanning set is simply:\\ 
    $\mathbf{x}$ = $\mathbf{0} = \begin{bmatrix}
        0 \\
        0 \\
        0
    \end{bmatrix}$\\
    $\left\{ 
        \begin{bmatrix}
            0 \\
            0 \\
            0
        \end{bmatrix}
    \right\}$
\end{center}

\end{document}
