\documentclass{report}

\usepackage{marvin} %found on https://github.com/marvinlinn/latexSetup. Add files to your local MiKTeX environment in (MiKTex Console > Settings > Directories > Add) and add the latexSetup folder.

\title{Winter 2022 MTH 261 Mini Test 4}
\author{Marvin Lin}
\date{March 2022}

\begin{document}

\maketitle

\section*{Question 1}
\begin{example}
    Let $A = \begin{bmatrix} 1 & 3 & -4 & 2 & -1 & 6 \\ 3 & 9 & -12 & 7 & 1 & 15 \\ -3 & -9 & 13 & -10 & 6 & -15 \\ 1 & 3 & -3 & -1 & 6 & 6 \end{bmatrix}$.
    
    \begin{enumerate}
    \item Find a basis for $Col A$.
    \item Find a basis for $Nul A$.
    \item Find a basis for $Row A$.
    \item Find a basis for $LNul A$.
    \item Find $dim Col A$, $dim Nul A$, $dim Col A^T$, $dim Nul A^T$.
    \end{enumerate}
\end{example}
\subsubsection*{Part 1}
Begin with the homogenous equation: $A\mathbf{x}=\mathbf{0}$\\
\begin{center}
    $\begin{bmatrix} 
        1 & 3 & -4 & 2 & -1 & 6 \\ 
        3 & 9 & -12 & 7 & 1 & 15 \\ 
        -3 & -9 & 13 & -10 & 6 & -15 \\ 
        1 & 3 & -3 & -1 & 6 & 6 
    \end{bmatrix}
    \begin{bmatrix} 
        x_1 \\ 
        x_2 \\ 
        x_3 \\
        x_4 \\
        x_5 \\ 
        x_6 
    \end{bmatrix}
    =
    \begin{bmatrix} 
        0 \\ 
        0 \\ 
        0 \\ 
        0 
    \end{bmatrix}
    $\\\vspace{3mm}
    $\begin{bmatrix} 
        1 & 3 & -4 & 2 & -1 & 6 & 0 \\ 
        3 & 9 & -12 & 7 & 1 & 15 & 0 \\ 
        -3 & -9 & 13 & -10 & 6 & -15 & 0 \\ 
        1 & 3 & -3 & -1 & 6 & 6 & 0 
    \end{bmatrix}
    $\\\vspace{3mm}
    $\sim \begin{bmatrix} 
        1 & 3 & -4 & 2 & -1 & 6 & 0 \\ 
        0 & 0 & 0 & 1 & 4 & -3 & 0 \\ 
        0 & 0 & 1 & -4 & 3 & 3 & 0 \\ 
        0 & 0 & 1 & -3 & 7 & 0 & 0 
    \end{bmatrix}
    $\\\vspace{3mm}
    $\sim \begin{bmatrix} 
        1 & 3 & -4 & 2 & -1 & 6 & 0 \\ 
        0 & 0 & 1 & -4 & 3 & 3 & 0 \\ 
        0 & 0 & 1 & -3 & 7 & 0 & 0 \\
        0 & 0 & 0 & 1 & 4 & -3 & 0 
    \end{bmatrix}
    $\\\vspace{3mm}
    $\sim \begin{bmatrix} 
        1 & 3 & -4 & 2 & -1 & 6 & 0 \\ 
        0 & 0 & 1 & -4 & 3 & 3 & 0 \\ 
        0 & 0 & 0 & 1 & 4 & -3 & 0 \\
        0 & 0 & 0 & 1 & 4 & -3 & 0 
    \end{bmatrix}
    $\\\vspace{3mm}
    $\sim \begin{bmatrix} 
        1 & 3 & -4 & 2 & -1 & 6 & 0 \\ 
        0 & 0 & 1 & -4 & 3 & 3 & 0 \\ 
        0 & 0 & 0 & 1 & 4 & -3 & 0 \\
        0 & 0 & 0 & 0 & 0 & 0 & 0 
    \end{bmatrix}
    $\\\vspace{3mm}
    $\sim \begin{bmatrix} 
        1 & 3 & 0 & 2 & 75 & -30 & 0 \\ 
        0 & 0 & 1 & 0 & 19 & -9 & 0 \\ 
        0 & 0 & 0 & 1 & 4 & -3 & 0 \\
        0 & 0 & 0 & 0 & 0 & 0 & 0 
    \end{bmatrix}
    $\\\vspace{3mm}
    $\sim \begin{bmatrix} 
        1 & 3 & 0 & 0 & 67 & -24 & 0 \\ 
        0 & 0 & 1 & 0 & 19 & -9 & 0 \\ 
        0 & 0 & 0 & 1 & 4 & -3 & 0 \\
        0 & 0 & 0 & 0 & 0 & 0 & 0 
    \end{bmatrix}
    $\\\vspace{3mm}
The Column space of A is simply the linear combination of all of the pivot columns of A. Therefore:\\\vspace{3mm}
    $Col A = 
    \{
        \begin{bmatrix}
            1 \\
            3 \\
            -3 \\
            1
        \end{bmatrix}
        ,
        \begin{bmatrix}
            -4 \\
            -12 \\
            13 \\
            -3
        \end{bmatrix}
        ,
        \begin{bmatrix}
            2 \\
            7 \\
            -10 \\
            -1
        \end{bmatrix}
    \}
    $
\end{center}
\subsubsection*{Part 2}
    We start with outcome of the homogenous equation to get the following:
\begin{center}
    $\sim \begin{bmatrix} 
        1 & 3 & 0 & 0 & 67 & -24 & 0 \\ 
        0 & 0 & 1 & 0 & 19 & -9 & 0 \\ 
        0 & 0 & 0 & 1 & 4 & -3 & 0 \\
        0 & 0 & 0 & 0 & 0 & 0 & 0 
    \end{bmatrix}
    $\\\vspace{3mm}
    $x_1 + 3x_2 + 67x_5 - 24x_6 = 0$\\
    $x_3 + 19x_5 - 9x_6 = 0$\\
    $x_4 + 4x_5 - 3x_6 = 0$
    \\\vspace{3mm}
    $x_1 = - 3x_2 - 67x_5 + 24x_6$\\
    $x_3 = - 19x_5 + 9x_6$\\
    $x_4 = - 4x_5 + 3x_6$
    \\\vspace{3mm}
    $
    \mathbf{x} = 
    \begin{bmatrix} 
        x_1 \\ 
        x_2 \\ 
        x_3 \\
        x_4 \\
        x_5 \\ 
        x_6 
    \end{bmatrix}
    =
    \begin{bmatrix} 
        -3x_2 \\ 
        1x_2 \\ 
        0x_2 \\
        0x_2 \\
        0x_2 \\ 
        0x_2 
    \end{bmatrix}
    +
    \begin{bmatrix} 
        -67x_5\\ 
        0x_5 \\ 
        -19x_5 \\
        -4x_5 \\
        1x_5 \\ 
        0x_5 
    \end{bmatrix}
    +
    \begin{bmatrix} 
        24x_6 \\ 
        0x_6 \\ 
        9x_6 \\
        3x_6 \\
        0x_6 \\ 
        1x_6 
    \end{bmatrix}
    =
    x_2\begin{bmatrix} 
        -3 \\ 
        1 \\ 
        0 \\
        0 \\
        0 \\ 
        0 
    \end{bmatrix}
    +
    x_5\begin{bmatrix} 
        -67 \\ 
        0 \\ 
        -19 \\
        -4 \\
        1 \\ 
        0 
    \end{bmatrix}
    +
    x_6\begin{bmatrix} 
        24 \\ 
        0 \\ 
        9 \\
        3 \\
        0 \\ 
        1 
    \end{bmatrix}
    $\\\vspace{3mm}
    So $NulA = span \{ 
        \begin{bmatrix} 
            -3 \\ 
            0 \\ 
            0 \\
            0 \\
            0 \\ 
            0 
        \end{bmatrix}
        ,
        \begin{bmatrix} 
            -67 \\ 
            0 \\ 
            -19 \\
            -4 \\
            0 \\ 
            0 
        \end{bmatrix}
        ,
        \begin{bmatrix} 
            24 \\ 
            0 \\ 
            9 \\
            3 \\
            0 \\ 
            0 
        \end{bmatrix}
        \}
        $
\end{center}
\subsubsection*{Part 3}
Once again, we begin with the homogenous equation: $A^T\mathbf{x}=\mathbf{0}$
\begin{center}
    $\begin{bmatrix} 
        1 & 3 & -3 & 1 \\ 
        3 & 9 & -9 & 3 \\ 
        -4 & -12 & 13 & -3 \\ 
        2 & 7 & -10 & -1 \\ 
        -1 & 1 & 6 & 6 \\ 
        6 & 15 & -15 & 6 
    \end{bmatrix}
    \begin{bmatrix} 
        x_1 \\ 
        x_2 \\ 
        x_3 \\
        x_4 
    \end{bmatrix}
    =
    \begin{bmatrix} 
        0 \\ 
        0 \\ 
        0 \\ 
        0 \\
        0 \\
        0 
    \end{bmatrix}
    $\\\vspace{3mm}
    $\begin{bmatrix} 
        1 & 3 & -3 & 1 & 0 \\ 
        3 & 9 & -9 & 3 & 0 \\ 
        -4 & -12 & 13 & -3 & 0 \\ 
        2 & 7 & -10 & -1 & 0 \\ 
        -1 & 1 & 6 & 6 & 0 \\ 
        6 & 15 & -15 & 6 & 0 
    \end{bmatrix}
    $\\\vspace{3mm}
    $\sim \begin{bmatrix} 
        1 & 3 & -3 & 1 & 0 \\ 
        0 & 0 & 0 & 0 & 0 \\ 
        0 & 0 & 1 & 1 & 0 \\ 
        0 & 1 & -4 & -3 & 0 \\ 
        0 & 4 & 3 & 7 & 0 \\ 
        0 & -3 & 3 & 0 & 0 
    \end{bmatrix}
    $\\\vspace{3mm}
    $\sim \begin{bmatrix} 
        1 & 3 & -3 & 1 & 0 \\ 
        0 & 1 & -4 & -3 & 0 \\ 
        0 & 4 & 3 & 7 & 0 \\ 
        0 & -3 & 3 & 0 & 0 \\
        0 & 0 & 1 & 1 & 0 \\
        0 & 0 & 0 & 0 & 0
    \end{bmatrix}
    $\\\vspace{3mm}
    $\sim \begin{bmatrix} 
        1 & 3 & -3 & 1 & 0 \\ 
        0 & 1 & -4 & -3 & 0 \\ 
        0 & 0 & 19 & 19 & 0 \\ 
        0 & 0 & -9 & -9 & 0 \\
        0 & 0 & 1 & 1 & 0 \\
        0 & 0 & 0 & 0 & 0
    \end{bmatrix}
    $\\\vspace{3mm}
    $\sim \begin{bmatrix} 
        1 & 3 & -3 & 1 & 0 \\ 
        0 & 1 & -4 & -3 & 0 \\
        0 & 0 & 1 & 1 & 0 \\ 
        0 & 0 & 19 & 19 & 0 \\ 
        0 & 0 & -9 & -9 & 0 \\
        0 & 0 & 0 & 0 & 0
    \end{bmatrix}
    $\\\vspace{3mm}
    $\sim \begin{bmatrix} 
        1 & 3 & -3 & 1 & 0 \\ 
        0 & 1 & -4 & -3 & 0 \\
        0 & 0 & 1 & 1 & 0 \\ 
        0 & 0 & 0 & 0 & 0 \\ 
        0 & 0 & 0 & 0 & 0 \\
        0 & 0 & 0 & 0 & 0
    \end{bmatrix}
    $\\\vspace{3mm}
    $\sim \begin{bmatrix} 
        1 & 3 & 0 & 4 & 0 \\ 
        0 & 1 & 0 & 1 & 0 \\
        0 & 0 & 1 & 1 & 0 \\ 
        0 & 0 & 0 & 0 & 0 \\ 
        0 & 0 & 0 & 0 & 0 \\
        0 & 0 & 0 & 0 & 0
    \end{bmatrix}
    $\\\vspace{3mm}
    $\sim \begin{bmatrix} 
        1 & 0 & 0 & 1 & 0 \\ 
        0 & 1 & 0 & 1 & 0 \\
        0 & 0 & 1 & 1 & 0 \\ 
        0 & 0 & 0 & 0 & 0 \\ 
        0 & 0 & 0 & 0 & 0 \\
        0 & 0 & 0 & 0 & 0
    \end{bmatrix}
    $\\\vspace{3mm}
    The row space is simply the column space of the transposed matrix. Therefore, the basis of the column space can be identified as:\\
    $ColA=
    \{
        \begin{bmatrix}
            1 \\
            3 \\
            -4 \\
            2 \\
            -1 \\
            6
        \end{bmatrix}
        ,
        \begin{bmatrix}
            3 \\
            9 \\
            -12 \\
            7 \\
            1 \\
            15
        \end{bmatrix}
        ,
        \begin{bmatrix}
            -3 \\
            -9 \\
            13 \\
            -10 \\
            6 \\
            -15
        \end{bmatrix}
    \}
    $
\end{center}
\subsubsection*{Part 4}
We can take the result of the RREF of $A^T$ to find the Left Null space, which is the null space of the transposed matrix:
\begin{center}
    $\sim \begin{bmatrix} 
        1 & 0 & 0 & 1 & 0 \\ 
        0 & 1 & 0 & 1 & 0 \\
        0 & 0 & 1 & 1 & 0 \\ 
        0 & 0 & 0 & 0 & 0 \\ 
        0 & 0 & 0 & 0 & 0 \\
        0 & 0 & 0 & 0 & 0
    \end{bmatrix}
    $\\\vspace{3mm}
    $x_1 + x_4 = 0$\\
    $x_2 + x_4 = 0$\\
    $x_3  + x_4 = 0$\\\vspace{3mm}
    $x_1 = - x_4$\\
    $x_2 = - x_4$\\
    $x_3 = - x_4$\\\vspace{3mm}
    LNul $A=$Nul$A^T= \{
        \begin{bmatrix}
            -1 \\
            -1 \\
            -1 \\
            1
        \end{bmatrix}
    \}$
\end{center}
\subsubsection*{Part 5}
The dimension of the spaces is simply the number of vectors in each space. Therefore:
\begin{center}
    dim Col $A = 3$\\
    dim Nul $A = 3$\\
    dim Row $A = 3$\\
    dim LNul $A = 1$\\  
\end{center}

\clearpage

\section*{Question 2}
\begin{example}
    Let
    $A = \begin{bmatrix} 1 & 1 & -1 \\ 1 & -1 & 1 \\ -1 & 1 & 1 \end{bmatrix} , \mathbf{u} = \begin{bmatrix} 1 \\ -2 \\ 1 \end{bmatrix} , \mathbf{v} = \begin{bmatrix} -2 \\ 0 \\ 2 \end{bmatrix} , \text{and}  \mathbf{w} = \begin{bmatrix} -1 \\ -1 \\ -1 \end{bmatrix}.$

    \begin{enumerate}
    \item Show that the characteristic polynomial for $A$ is $-\lambda^3 + \lambda^2 + 4\lambda - 4$.
    \item Find the eigenvalues of $A$.
    \item Find a basis for the eigenspace of each eigenvalue.
    \item Find $A\mathbf{u}, A\mathbf{v},$ and $A\mathbf{w}$ without using a matrix-vector product.
    \end{enumerate}
\end{example}
\subsubsection*{Part 1}
\begin{center}
    $det(A-\lambda I)=0$\\\vspace{3mm}
    $det(
        \begin{bmatrix}
            1 & 1 & -1 \\ 
            1 & -1 & 1 \\ 
            -1 & 1 & 1 
        \end{bmatrix}
        -
        \begin{bmatrix}
            \lambda & 0 & 0 \\
            0 & \lambda & 0 \\
            0 & 0 & \lambda
        \end{bmatrix}
    )=0
    $\\\vspace{3mm}
    $\begin{vmatrix}
            1-\lambda & 1 & -1 \\ 
            1 & -1-\lambda & 1 \\ 
            -1 & 1 & 1-\lambda
        \end{vmatrix}
        =0
    $\\\vspace{3mm}
    $(1-\lambda)((-1-\lambda)(1-\lambda)-1)-
    (1)((1)(1-\lambda)+1)+
    (-1)((1)(1)-(-1-\lambda)(-1))
    =0
    $\\\vspace{3mm}
    $(1-\lambda)((-1-\lambda)(1-\lambda)-1)-
    (2+\lambda)-
    (-\lambda)
    =0
    $\\\vspace{3mm}
    $(1-\lambda)((-1-\lambda)(1-\lambda)-1)-2+2\lambda
    =0
    $\\\vspace{3mm}
    $(1-\lambda)(-1+\lambda-\lambda+\lambda^2-1)-2+2\lambda
    =0
    $\\\vspace{3mm}
    $(1-\lambda)(\lambda^2-2)-2+2\lambda
    =0
    $\\\vspace{3mm}
    $\lambda^2-2-\lambda^3+2\lambda-2+2\lambda
    =0
    $\\\vspace{3mm}
    $-\lambda^3+\lambda^2+4\lambda-4
    =0
    $\\\vspace{3mm}
    Therefore, the characteristic equation is $-\lambda^3+\lambda^2+4\lambda-4$
\end{center}
\subsubsection*{Part 2}
Simply factor the equation to get:
\begin{center}
    $-(\lambda-1)(\lambda+2)(\lambda-2)=0$
\end{center}
Therefore, the eignvalues are $\lambda = 1, -2, 2$
\subsubsection*{Part 3}
\begin{center}
$\mathbb{B}_{\lambda=1}=
nul\begin{bmatrix}
    1-1 & 1 & -1 \\ 
    1 & -1-1 & 1 \\ 
    -1 & 1 & 1-1
\end{bmatrix}=
nul\begin{bmatrix}
    0 & 1 & -1 \\ 
    1 & -2 & 1 \\ 
    -1 & 1 & 0
\end{bmatrix}$\\\vspace{3mm}
$\sim \begin{bmatrix}
    1 & -2 & 1 & 0 \\ 
    -1 & 1 & 0 & 0 \\
    0 & 1 & -1 & 0 
\end{bmatrix}$\\\vspace{3mm}
$\sim \begin{bmatrix}
    1 & -2 & 1 & 0 \\ 
    0 & -1 & 1 & 0 \\
    0 & 1 & -1 & 0 
\end{bmatrix}$\\\vspace{3mm}
$\sim \begin{bmatrix}
    1 & -2 & 1 & 0 \\ 
    0 & 1 & -1 & 0 \\
    0 & 0 & 0 & 0 
\end{bmatrix}$\\\vspace{3mm}
$\sim \begin{bmatrix}
    1 & 0 & -1 & 0 \\ 
    0 & 1 & -1 & 0 \\
    0 & 0 & 0 & 0 
\end{bmatrix}$\\\vspace{3mm}
$x_1-x_3=0$\\
$x_2-x_3=0$\\
$x_3=x_3$\\\vspace{3mm}
$x_1=x_3$\\
$x_2=x_3$
$x_3=x_3$\\\vspace{3mm}
Therefore, $\mathbb{B}_{\lambda=1}=\begin{bmatrix}
    1 \\
    1 \\
    1
\end{bmatrix}$.\vspace{3mm}

$\mathbb{B}_{\lambda=2}=
    nul\begin{bmatrix}
        1-2 & 1 & -1 \\ 
        1 & -1-2 & 1 \\ 
        -1 & 1 & 1-2
    \end{bmatrix}=
    nul\begin{bmatrix}
        -1 & 1 & -1 \\ 
        1 & -3 & 1 \\ 
        -1 & 1 & -1
    \end{bmatrix}$\\\vspace{3mm}
    $\sim \begin{bmatrix}
        1 & -3 & 1 & 0 \\ 
        -1 & 1 & -1 & 0 \\
        -1 & 1 & -1 & 0 
    \end{bmatrix}$\\\vspace{3mm}
    $\sim \begin{bmatrix}
        1 & -3 & 1 & 0 \\ 
        -1 & 1 & -1 & 0 \\
        0 & 0 & 0 & 0 
    \end{bmatrix}$\\\vspace{3mm}
    $\sim \begin{bmatrix}
        1 & -3 & 1 & 0 \\ 
        0 & -2 & 0 & 0 \\
        0 & 0 & 0 & 0 
    \end{bmatrix}$\\\vspace{3mm}
    $\sim \begin{bmatrix}
        1 & -3 & 1 & 0 \\ 
        0 & 1 & 0 & 0 \\
        0 & 0 & 0 & 0 
    \end{bmatrix}$\\\vspace{3mm}
    $\sim \begin{bmatrix}
        1 & 0 & 1 & 0 \\ 
        0 & 1 & 0 & 0 \\
        0 & 0 & 0 & 0 
    \end{bmatrix}$\\\vspace{3mm}
    $x_1+x_3=0$\\
    $x_2=0$
    $x_3=x_3$\\\vspace{3mm}
    $x_1=-x_3$\\
    $x_2=0$
    $x_3=x_3$\\\vspace{3mm}
    Therefore, $\mathbb{B}_{\lambda=2}=\begin{bmatrix}
        -1 \\
        0 \\
        1
    \end{bmatrix}$.\vspace{3mm}

    $\mathbb{B}_{\lambda=-2}=
    nul\begin{bmatrix}
        1--2 & 1 & -1 \\ 
        1 & -1--2 & 1 \\ 
        -1 & 1 & 1--2
    \end{bmatrix}=
    nul\begin{bmatrix}
        3 & 1 & -1 \\ 
        1 & 1 & 1 \\ 
        -1 & 1 & 3
    \end{bmatrix}$\\\vspace{3mm}
    $\sim \begin{bmatrix}
        1 & 1 & 1 & 0 \\ 
        3 & 1 & -1 & 0 \\
        -1 & 1 & 3 & 0
    \end{bmatrix}$\\\vspace{3mm}
    $\sim \begin{bmatrix}
        1 & 1 & 1 & 0 \\ 
        0 & -2 & -4 & 0 \\
        0 & 2 & 4 & 0
    \end{bmatrix}$\\\vspace{3mm}
    $\sim \begin{bmatrix}
        1 & 1 & 1 & 0 \\ 
        0 & -2 & -4 & 0 \\
        0 & 0 & 0 & 0
    \end{bmatrix}$\\\vspace{3mm}
    $\sim \begin{bmatrix}
        1 & 1 & 1 & 0 \\ 
        0 & 1 & 2 & 0 \\
        0 & 0 & 0 & 0
    \end{bmatrix}$\\\vspace{3mm}
    $\sim \begin{bmatrix}
        1 & 0 & -1 & 0 \\ 
        0 & 1 & 2 & 0 \\
        0 & 0 & 0 & 0
    \end{bmatrix}$\\\vspace{3mm}
    $x_1-x_3=0$\\
    $x_2+2x_3=0$
    $x_3=x_3$\\\vspace{3mm}
    $x_1=-2x_3$\\
    $x_2=0$
    $x_3=x_3$\\\vspace{3mm}
    Therefore, $\mathbb{B}_{\lambda=-2}=\begin{bmatrix}
        1 \\
        -2 \\
        1
    \end{bmatrix}$.\vspace{3mm}
\end{center}
\subsubsection*{Part 4}
Since we can simply use the linear combinations of the basis of the eigenspace of each eigenvalue (the eigenvectors), we can easily calculate the following:
\begin{center}
    $A\mathbf{u}=A\begin{bmatrix}
        1 \\
        -2 \\
        1 \\
    \end{bmatrix} = 
    -2\begin{bmatrix}
        1 \\
        -2 \\
        1 \\
    \end{bmatrix}=
    \begin{bmatrix}
        -2 \\
        4 \\
        -2 \\
    \end{bmatrix}$\\
    $A\mathbf{v}=A\begin{bmatrix}
        -2 \\
        0 \\
        2 \\
    \end{bmatrix} = 
    2(A\begin{bmatrix}
        -1 \\
        0 \\
        1 \\
    \end{bmatrix}) = 
    2(2\begin{bmatrix}
        -1 \\
        0 \\
        1 \\
    \end{bmatrix})=
    \begin{bmatrix}
        -4 \\
        0 \\
        4 \\
    \end{bmatrix}$\\
    $A\mathbf{w}=A\begin{bmatrix}
        -1 \\
        -1 \\
        -1 \\
    \end{bmatrix} = 
    -1(A\begin{bmatrix}
        1 \\
        1 \\
        1 \\
    \end{bmatrix}) = 
    -1(1\begin{bmatrix}
        1 \\
        1 \\
        1 \\
    \end{bmatrix})=
    \begin{bmatrix}
        -1 \\
        -1 \\
        -1 \\
    \end{bmatrix}$\\
\end{center}
\end{document}